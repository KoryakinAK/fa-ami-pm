\documentclass[12pt,a4paper]{article}
\usepackage[14pt]{extsizes} 
\usepackage[utf8]{inputenc}
\usepackage{amsmath}
\usepackage{amsfonts}
\usepackage{amssymb}
\usepackage{cmap}
% for fonts
    \usepackage[T2A, T1]{fontenc}
    \usepackage[english, russian]{babel}
    \usepackage{fontspec}
    \defaultfontfeatures{Ligatures=TeX,Renderer=Basic}
    \setmainfont[Ligatures={TeX, Historic}]{Times New Roman}
    \setsansfont{Times New Roman}
    \setmonofont{Courier New}
\usepackage{pgfplots} % plot
\pgfplotsset{compat=1.15}
\usepackage{graphicx}
\usepackage[left=2cm,right=2cm,top=2cm,bottom=2cm]{geometry}
\author{Аверьянов Т.С.}
\begin{document}
\begin{center}
\section*{Третий принцип спецификации экономических моделей: отражение в модели, факторы времени}
\end{center}
\begin{center}
\textbf{План}
\end{center}
\begin{enumerate}
\item Спецификация динамической модели спроса-предложения на конкурентом рынке. Типы переменных в динамических моделях.
\item Трансформация динамической модели к приведенной форме. Предельные величины в экономике. 
\item ДЗ, защита ДЗ
\end{enumerate}

Мы обсудили два принципа спецификации эконометрических моделей и две формы; обсуждение провели на примере простейшей модели спроса-предложения на конкурентом рынке.

3 эндогенные переменные: спрос, предложение и цена; 1 эндогенная: x.

В этой модели о взаимосвязи эндогенных  и экзогенных переменных по существу заложено предположение, что эндогенные переменные практически мгновенно реагируют на уровень душевого дохода потребителя и уровень предложения.

Между тем уровень предложения блага. В текущем периоде обладает определённой инерцией по отношению к изменению цены блага. Точнее уровень предложения в текущем периоде лучше объясняется ценой блага в предшествующем периоде, потому что производителю необходимо время для перестройки производства. Подчеркнём что в этом утверждении содержится фактор времени и мы обязаны в процессе записи математическим языком данного утверждения различать цену блага (сметаны) в текущем периоде и в предшествующем.

В текущем периоде $p_t$, обозначим цену блага в предшествующем $p_{t-1}$ (лагавой ценой). Таким образом мы можем сформулировать закон: Уровень предложения объясняется ($y_t^s = y_t^s(p_{t-1})\uparrow$). Напротив уровень блага мгновенно реагирует на уровень дохода потребителя $y_t^d = y_t^d(p_t, x)\downarrow \uparrow$. Закон формирования рыночной цены в текущем периоде сохраняется и в данном случае: $p_t$ (цена в текущем периоде) формируется при балансе текущего спроса и текущего предложения, требутся составить модель которая позволяет объяснять уровень спроса, предложения душевым доход в текущем периоде. Кроме известной величиной является лаговая цена блага. Таким образом в данной задаче с уточнённым законом предложения будут присутствовать две объясняющие велечины:
\begin{itemize}
\item Текущей эндогенной перменной
\item Лагавой эндогенной переменной
\end{itemize}
\begin{equation}
(y_t^s, y_t^s, p_t)
\end{equation}
\begin{equation}
(x_t, p_{t - 1})
\end{equation}
\begin{equation}
\begin{cases}
p_t, p_{t-1} \\
y_t^s = y_t^s(p_{t-1})\uparrow \\
y_t^d = y_t^d(p_t, x)\downarrow \uparrow
\end{cases}
\end{equation}
\begin{equation}
\begin{cases}
y_t^d = a_0 + a_1 p_t + x_t - \text{(прос. лин. модель спроса)}, \\
a_1 < 0, a_2 > 0
\end{cases}
\end{equation}
\begin{equation}
\begin{cases}
y_t^s = b_0 + b_1 p_{t - 1} - \text{(прос. лин. модель предложения)}, \\
b_0 > 0
\end{cases}
\end{equation}
\begin{equation}
\begin{cases}
y_t^s = y_t^d
\end{cases}
\end{equation}

Три уравнения образуют структурную форму простейшей экономической модели нормально ценного блага на конкурентном рынке.

\textbf{Итог.} Для отражения в модели фактора времения все переменные модели датируются, т.е. привязываются ко времени и в итоге возникает спецификация динамической модели. Подчеркнём, что в набор обясняющих переменных (2) могут входить лаговые эндогенные переменные.

\textbf{Задача.} Трасформировать модель выше к преведённой форме:
\begin{equation}
\begin{cases}
y_t^d = y_t^d(p_{t-1}, x_t) \\
y_t^s = y_t^s(p_{t-1}, x_t) \\
p_t = p_t(p_{t-1}, x_t)
\end{cases}
\end{equation}

Первый шаг будет точно такой же как, как в лекции 1. Решаем методом подставновки Гаусса.
$$a_0 + a_1 p_t + x_t = b_0 + b_1 p_{t - 1}$$
\begin{equation}
p = \frac{b_0 - a_0}{a_1} + \frac{b_1}{a_1}p_{t-1} - \frac{a_2}{a_1} x_t
\end{equation}
Уравнение (8) преведённая форма текущей цены.

Второй шаг приведённая форма предложения уже содержится в структурной форме модели:
$$y_t^s = b_0 + b_1 p_{t - 1}$$

В силу (6) уравнения:
$$y_t^d = b_0 + b_1 p_{t - 1}$$

Простейшая модель спроса и предложения:
\begin{equation}
\begin{cases}
p = \frac{b_0 - a_0}{a_1} + \frac{b_1}{a_1}p_{t-1} - \frac{a_2}{a_1} x_t \\
y_t^s = b_0 + b_1 p_{t - 1} \\
y_t^d = b_0 + b_1 p_{t - 1}
\end{cases}
\end{equation}
Сопоставляя приведённые формы статической модели спроса и предложения (Семинар 1) и динамической модели (9) мы видим, что это совершенно различные модели.

\section*{Предельные величины в экономике}
Вернёмся к приведённой форме (9) и обозначим:
$$p = \alpha_0 + \alpha_1 p_{t-1} - \alpha_2 x_t$$
Наша цель выяснить экономический смысл $\alpha_1, \alpha_2$. Предположим, что $p_{t - 1}, x_t + \delta x_t$, тогда в силу уравнения (8): $p_t + \delta p_t = \alpha_0 + \alpha_1 p_{t-1} + \alpha_2(x_t + \delta x_t) (*)$, вычитая уравнения получим $\delta p_t = \alpha_2 \delta x_t (**)$
Таким образом $\alpha_2$ -- это ответ на $x_t$. Такую комбинацию называют предельным значением $p_t$ по объясняющей переменной $x_t$. 

Добавим, что $\alpha_2$ можно расчитать по правилу: нужно взять производную.

\textbf{Задача}. Вычислить $\alpha_2$ и дать экономическую интерпретацию. Рассматриваю знаки коэффициентов в структурной форме, мы убеждаемся, что $\alpha_2 > 0$.

\framebox[1.1\width]{Д/з} Уточнить динамический закон предложения, согласно уточнённому закону
$$y_t^s = y_t^s(p_{t-1}, p_{t-1}^{m}) \downarrow \uparrow $$
Лагаваю цену сырья интерпретировать, как лаговую экзогенную переменную. Трансформировать такую динамическую модель к преведённой форме. И выяснисть знак у текущего спроса по лаговой цене сырья $\displaystyle{\frac{d y_t^d}{d p_{t-1}^{(m)}}}[10pt]$
\end{document}
