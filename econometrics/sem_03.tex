\documentclass[12pt,a4paper]{article}
\usepackage[14pt]{extsizes} 
\usepackage[utf8]{inputenc}
\usepackage{amsmath}
\usepackage{amsfonts}
\usepackage{amssymb}
\usepackage{cmap}
% for fonts
    \usepackage[T2A, T1]{fontenc}
    \usepackage[english, russian]{babel}
    \usepackage{fontspec}
    \defaultfontfeatures{Ligatures=TeX,Renderer=Basic}
    \setmainfont[Ligatures={TeX, Historic}]{Times New Roman}
    \setsansfont{Times New Roman}
    \setmonofont{Courier New}
% mathcha
\usepackage{tikz}
\usepackage{mathdots}
\usepackage{yhmath}
\usepackage{cancel}
\usepackage{color}
\usepackage{siunitx}
\usepackage{array}
\usepackage{multirow}
\usepackage{amssymb}
\usepackage{gensymb}
\usepackage{tabularx}
\usepackage{booktabs}
\usetikzlibrary{fadings}
% mathcha
\usepackage{pgfplots} % plot
\usepackage{float} % for H at figure
\usepackage{cases}
\pgfplotsset{compat=1.15}
\usepackage{graphicx}
\usepackage[left=2cm,right=2cm,top=2cm,bottom=2cm]{geometry}
\author{GH-TIMe}
\begin{document}
\begin{enumerate}
\item Аналитическое описание диаграммы рассеянья (Лаговый доход -- текущее потребление домохозяйств). Четвёртый принцип спецификации эконометрических моделей;
\item Обсуждение домашних заданий;
\end{enumerate}

На двух предшествующих занятиях обсудили три принципа спецификации эконометрических моделей:
\begin{enumerate}
\item Словестный (вербальный) модель возникает в результате записи мат. языком взаимосвязей
\item Количество уравнений моделей равно количеству текущих эндогенных переменных
\item Отражения фактрора време, датирование всех переменных моделей
\end{enumerate}

На сегодняшнем занятии мы рассмотрим отражение её не полного соответсветсвия изучаемому объекту, т.е. отражение в спецификации модели влияние на текущие эндогенные переменные неучтённых факторов. В качестве изучаемого объекта мы выберем экономику россии, изучать этот объект мы будем превлекая модель Самуэльсона-Хикса делового цикла экономики. Состояние экономики в текущем периоде $t$ описывается основными макроэкономическими переменными: $Y_t$ -- уровень ВВП страны $C_t$ -- уровень расхода домохозяйств на потребление, $I_t$ -- объём инвестиций в экономику (накопление капитала), $G_t$ -- уровень государственных расходов.

Сейчас требуется составить спецификацию модели позволяющей объяснять переменные $\displaystyle Y_{t} ,\ C_{t} ,\ I_{t} ,\ G_{t}$ их лаговыми значениями. При составлении спецификации нужно учесть следующие экономические утверждения (стр. 38 задача 2.5):
\begin{enumerate}
\item Текущий расходы домохозяйств на конечное потребление $\displaystyle C_{t}$ объясняются уровнем ВВП в предыдущем периоде, возрастая вместе с ним;
\item Величина инвестиций в текущем периоде (валовое накопление капитала) $\displaystyle I_{t}$ объясняется приростом ВВП за предыдущий период (прирост ВВП за предшествующий период - это разность $\displaystyle \triangle Y_{t-1} \ =\ Y_{t-1\ } \ -\ Y_{t-2}$); с ростом $\displaystyle \triangle Y_{t-1}$ увеличивается и $\displaystyle I_{t}$;
\item Государственные расходы $\displaystyle G_{t}$ возрастают с постоянным темпом роста;
\item Текущее значение ВВП $\displaystyle Y_{t}$ есть сумма текущих уровней потребления домохозяйств, инвестиций и государственных расходов (основное тождество системы национальных счетов)
\end{enumerate}


\begin{equation*}
\begin{cases}
C_{t} \ =\ a_{0} \ +\ a_{1} \ \cdot Y_{t} ; & a_{1} \  >\ 0\\
I_{t} \ =\ b_{0} \ +b_{1} \ \cdot \ ( Y_{t-1} \ -\ Y_{t-2}) ; & b_{1} \  >0\\
G_{t} \ =\ g\ \cdot \ G_{t-1} ; & g >1\\
Y_{t} \ =\ C_{t} \ +\ I_{t} \ +\ G_{t} ; & 
\end{cases}
\end{equation*}

\begin{gather*}
\text{Объсняющие\ переменные}:\overrightarrow{x_{t}} =( Y_{t-1} ,Y_{t-2} ,Y_{t-3})\\
\text{Объясняемые\ переменные}:\overrightarrow{y_t} \ =\ ( Y_{t} ,\ C_{t} ,\ I_{t} ,\ G_{t})
\end{gather*}

	Зададимся следующим вопросом. Согласуется ли данная модель с изучаемым объектом (экономикой россии)? Другими словами соответсвует ли эта модель реальной статистической информации из реальных счетов россии? Чтобы ответить на этот вопрос проведём следующее иследование:
\begin{enumerate}
\item Выберем в моделе Самуэльсона-Хикса уравнение текущих расходов домохозяйств $\displaystyle C_{t} \ =\ a_{0} \ +\ a_{1} \ \cdot Y_{t}$ и проведем соответсвие этой модели статистической информации из системы национальных счетов России; Это информация содержится в файле (на почте) элементы использования ВВП России; Эту информации можно собрать на сайте www.gks.ru/;
\item В Excel построим график, откладывая вдоль горизонтальной оси лаговые уровни ВВП, в доль вертикальной оси соответсвующиe уровни домохозяйств $\displaystyle C_{t}$. Если модель Самуэльсона-Хикса в полной мере соответсвует реальным данным, то построенный график (построенное множество точек) будет предствалять собой восходящую прямую. Если же график окажется иным, то будет означать, что по крайней мере первое уравнение не соответсвует реально статистике.
\begin{figure}[H]
\begin{center}
\tikzset{every picture/.style={line width=0.75pt}} %set default line width to 0.75pt        

\begin{tikzpicture}[x=0.75pt,y=0.75pt,yscale=-1,xscale=1]
%uncomment if require: \path (0,312.8249969482422); %set diagram left start at 0, and has height of 312.8249969482422

%Shape: Axis 2D [id:dp8743509274123296] 
\draw  (50,251.98) -- (384.3,251.98)(83.43,68) -- (83.43,272.43) (377.3,246.98) -- (384.3,251.98) -- (377.3,256.98) (78.43,75) -- (83.43,68) -- (88.43,75)  ;
%Straight Lines [id:da44467330718849607] 
\draw    (81.3,206.43) -- (347.3,123.43) ;

%Straight Lines [id:da6441726223509632] 
\draw    (140.3,187.43) -- (140,230) ;
\draw [shift={(140,230)}, rotate = 90.4] [color={rgb, 255:red, 0; green, 0; blue, 0 }  ][fill={rgb, 255:red, 0; green, 0; blue, 0 }  ][line width=0.75]      (0, 0) circle [x radius= 3.35, y radius= 3.35]   ;

%Straight Lines [id:da45408138515545904] 
\draw    (179.3,134.43) -- (179,176) ;

\draw [shift={(179.3,134.43)}, rotate = 90.41] [color={rgb, 255:red, 0; green, 0; blue, 0 }  ][fill={rgb, 255:red, 0; green, 0; blue, 0 }  ][line width=0.75]      (0, 0) circle [x radius= 3.35, y radius= 3.35]   ;

%Straight Lines [id:da3675690211779734] 
\draw    (229.3,159.43) -- (229,201) ;
\draw [shift={(229,201)}, rotate = 90.41] [color={rgb, 255:red, 0; green, 0; blue, 0 }  ][fill={rgb, 255:red, 0; green, 0; blue, 0 }  ][line width=0.75]      (0, 0) circle [x radius= 3.35, y radius= 3.35]   ;

%Straight Lines [id:da8370177312160028] 
\draw    (259.3,131.43) -- (259.3,151.43) ;

\draw [shift={(259.3,131.43)}, rotate = 90] [color={rgb, 255:red, 0; green, 0; blue, 0 }  ][fill={rgb, 255:red, 0; green, 0; blue, 0 }  ][line width=0.75]      (0, 0) circle [x radius= 3.35, y radius= 3.35]   ;
%Straight Lines [id:da37313575108824626] 
\draw    (290.3,99.43) -- (290,141) ;

\draw [shift={(290.3,99.43)}, rotate = 90.41] [color={rgb, 255:red, 0; green, 0; blue, 0 }  ][fill={rgb, 255:red, 0; green, 0; blue, 0 }  ][line width=0.75]      (0, 0) circle [x radius= 3.35, y radius= 3.35]   ;
%Straight Lines [id:da6953659916864847] 
\draw    (330.3,128.43) -- (330.3,152.43) ;
\draw [shift={(330.3,152.43)}, rotate = 90] [color={rgb, 255:red, 0; green, 0; blue, 0 }  ][fill={rgb, 255:red, 0; green, 0; blue, 0 }  ][line width=0.75]      (0, 0) circle [x radius= 3.35, y radius= 3.35]   ;


% Text Node
\draw (112,213) node  [align=left] {$\displaystyle u_{1}$};
% Text Node
\draw (149,146) node  [align=left] {$\displaystyle u_{2}$};
% Text Node
\draw (210,194) node  [align=left] {$\displaystyle u_{3}$};
% Text Node
\draw (244,122) node  [align=left] {$\displaystyle u_{4}$};
% Text Node
\draw (312,101) node  [align=left] {$\displaystyle u_{5}$};
% Text Node
\draw (363,153) node  [align=left] {$\displaystyle u_{6}$};
% Text Node
\draw (99,75) node  [align=left] {$\displaystyle C$};
% Text Node
\draw (363,233) node  [align=left] {$\displaystyle Y$};


\end{tikzpicture}
\end{center}
\caption{Уравнение потребления}
\end{figure}
\item Рассматривая данный график мы констатируем, что точки не распределились на воходящей прямой и это значит, что модель Самуэльсона-Хикса не в полной мере соответсвует изчуемому объекту; Основная причина неполного соответствие влияние на уровень потребления домохозяйств неучтённых факторов.
\end{enumerate}

	$\displaystyle \boxed{\text{ДЗ}}$ осуществить аналогичное исследование модели Самуэльсона-Хикса для уравнения инвестиций и государственных расходов. 

	Аналитическое описание диаграммы рассеивания. Построенный график принято называть диаграммой рассеивания. Нам предстоит аналитически описать эту диаграмму. Рассматривая эту диаграмму мы можем сделать следующие выводы:
\begin{enumerate}
\item Точки реальных данных разметились вдоль восходящей прямой;
\item Точки реальных данных хаотично разместились вдоль восходящей прямой, то выше, то ниже, то ближе, то дальше.
\end{enumerate}

	Вот аналитическая запись данной диаграммы:
	
\begin{equation}
\begin{cases}
C_{t} \ =a_{0} \ +a_{1} Y_{t-1} \ +\ U_{t} ;\\
E( U_{t}) \ =\ 0;Var( U_{t}) = \sigma^{2}_{u}
\end{cases}
\end{equation}

	Символом $\displaystyle U_{t}$ мы обозначили переменную величину, которая хаотично принимает то положительные, то отрицательные значения рассеяные вокруг 0. Символов $\displaystyle E( U_{t})$ мы обозначили среднее значение, символом $\displaystyle Var( U_{t})$ диспресию. В силу хаотичности появления её значений экономисты называют \textit{случайным возмущением}. Физики и в технических приложениях такие величины называются невязками или ошибками модели.

\textbf{Итог: }Выражение (1) более адекватно описывает лаговое потребление в стране и текущего потреблени домохозяйств.

	\textbf{Общий вывод:} для отражения на текущие эндогенные переменные неучтённых факторов в правых частях поведенческих моделей дискриптивных моделей включаются случайные возмущения. В этом и состоит четвёртый принцип спецификации экномметрических моделей.

	$\displaystyle \boxed{\text{ДЗ}}$ построить графики в Excel.
\end{document}