\documentclass[12pt,a4paper]{article}
\usepackage[14pt]{extsizes} 
\usepackage[utf8]{inputenc}
\usepackage{amsmath}
\usepackage{amsfonts}
\usepackage{amssymb}
\usepackage{cmap}
% for fonts
    \usepackage[T2A, T1]{fontenc}
    \usepackage[english, russian]{babel}
    \usepackage{fontspec}
    \defaultfontfeatures{Ligatures=TeX,Renderer=Basic}
    \setmainfont[Ligatures={TeX, Historic}]{Times New Roman}
    \setsansfont{Times New Roman}
    \setmonofont{Courier New}
%plot
\usepackage{pgf,tikz,pgfplots}
\pgfplotsset{compat=1.15}
\usepackage{mathrsfs}
\usetikzlibrary{arrows}
\pagestyle{empty}
%plot
\usepackage{float}% for \begin{figure}[H]
\usepackage{cases}
\usepackage{graphicx}
\usepackage[left=2cm,right=2cm,top=2cm,bottom=2cm]{geometry}
\author{GH-TIMe}
\begin{document}

\begin{center}
\section*{Эконометрика её задача и методы (Лекция №1)}
\end{center}

\framebox[1.1\width]{Д/з} выполняется на отдельных листках.
\begin{center}
\textbf{План}
\end{center}
\begin{enumerate}
\item Структура экономических задач;
\item Эконометрика, её задача и методы;
\item Первый принцип спецификации эконометрических моделей и эконометрическая теория;
\item Второй принцип спецификации моделей и алгебра
\item Третий принцип спецификации эконометрических моделей отражение фактора времени;
\item Приведённая форма модели, как инструмент анализа экономического объекта;
\end{enumerate}
\textbf{Структра эконометрических задач}
\begin{enumerate}
\item Исходные даннные (значения известны): \begin{equation}
x_1, x_2, \cdots, x_n
\end{equation}
\item Искомые неизвестные: \begin{equation}
y_1, y_2, \cdots, y_m
\end{equation}
\item Взаимосвязи велечин (1) и (2);
\end{enumerate}

В эконометрических задачах взаимосвязи существуют объективно и как правило инткитиано ощущаются. Объективный характер взаимосвязи позволяет приблизительно вычислить эндогенные переменные (2)

\textbf{Задача Кейнса}

\begin{itemize}
\item Исходные данные в задаче Кейнса считается $I$ - объём инвестиций в экономику страны на заданном отрезке времени;
\item Искомые неизвестные:
\begin{enumerate}
\item $Y$ -- уровень дохода в стране, в том же периоде (ВВП);
\item $C$ -- велечина совокупного потребления;
\end{enumerate}
\item Взаимосвязи велечин $(I, Y, C)$ отражены в следующих утверждения эконмической теории:
\begin{enumerate}
\item Доход $Y$ образует потребительские (государственные и индивидуальные) расходы $C$ и инвестиционные расходы $I$;
\item Уровень потребления $C$ объясняется величиной дохода $Y$;
\item Каждая дополнительная еденица дохода, $\triangle Y = 1$ потребляется, как правило, не полностью: часть её идёт на инсветиции;
\end{enumerate}
\end{itemize}

В любой математической задачае можно выделить три принципа спецификации. Взаимосвязи записанные математическим языком образуют математическую модель данной задачи.

\textbf{Эконометрика её задачи и методы}

\begin{itemize}
\item \underline{Эконометрика} -- прикладная математическая дисциплина, в которой изучаются конкретные количесвенные взаимосвязи объектов и процессов;
\item \underline{Задача} эконметрики заключается в объяснении(прогнозе или приближённом вычислении) искомых количественных характеристик (2) по известным значениям (1) каких-то других количественных характеристик этого объекта (задачи или процесса). Приближённые значения $(\widetilde{y_1}, \widetilde{y_2}, \cdots, \widetilde{y_m})$;
\end{itemize}

\textbf{Эконметрика её задача и метод эконометрики}

Метод решения задачи эконометрики состоит в предварительном построении упрощённой схемы изучаемого объекта (задачи или процесса), $F(\vec{y}, \vec{x}) = 0$, составленной математическим языком и именуемой эконометрической моделью, а затем в вычислении по этой модели приближённых значений неизвестных (2), $\vec{y} = f(\vec{x})$.

Прокомментируем $F(\vec{y}, \vec{x}) = 0$ и $\vec{y} = f(\vec{x})$. Символом $F$ обозначены взаимосвязи (1) и (2). $\vec{y}$ -- это весь набор величин (2). $\vec{x}$ -- это весь набор исходных величин (1). Выражение\begin{equation}
\vec{y} = f(\vec{x})
\end{equation}
, где каждая искомая велечина (2) выражена только через известные велечины (1).
\begin{itemize}
\item (1) -- экзогенные переменные;
\item (2) -- эндогенные переменные;
\item (3) -- приведённая форма;
\end{itemize}
Задача эконометрики состоит в поиске приближённых значений на основании известных характеристик. Метод решения этой задачи заключается в предварительном построении модели и вычислении (2).

\textbf{Первый принцип спецификации модели и экономическая теория}

Приступаем к изучению принципов/приёмов, которыми пользуются экономисты в процессе построения.

Модель $F(\vec{y}, \vec{x}) = 0$ возникает в итоге трансляции на математический язык экономических утверждений о взаимосвязях исходных данных (1) и искомых неизвестных (2) объекта (процесса или задачи). Результат трансляции неоднозначен (возможны варианты!). Стараются привлекать линейные функции, так как они простые.

\underline{Пример}.(Задача Кейнса)
\begin{equation}
\begin{cases}
Y = C + I; \\
C = a_0 + a_1 Y; \\
0 < a_1 < 1
\end{cases}
\end{equation}

\framebox[1.1\width]{Д/з} 2. и 3. записать математическим языком.

На следующем рисунке показан график:

\begin{figure}[H]
\begin{center}
\begin{tikzpicture}[line cap=round,line join=round,>=triangle 45,x=1.0cm,y=1.0cm]
\begin{axis}[
x=1.0cm,y=1.0cm,
xlabel={$Y$},
ylabel={$C$},
axis lines=middle,
xticklabels={,,},
yticklabels={0, 1, $a_0$},
ymajorgrids=true,
xmajorgrids=true,
xmin=0.0,
xmax=8.0,
ymin=0.0,
ymax=5.0,
xtick={0.0,1.0,...,7.0}]
\clip(0.,0.) rectangle (7.,4.);
\draw [line width=2.pt,domain=0.0:7.0] plot(\x,{(--5.122704733283244--1.8334184823441002*\x)/5.122704733283244});
\draw [line width=2.pt] (1.9993484898008496,1.71556778395429)-- (3.409706987227647,1.736498873027797);
\draw [line width=2.pt] (3.409706987227647,1.736498873027797)-- (3.4135239214209543,2.2217018495707332);
\draw (2,1.78) node[anchor=north west] {$\triangle Y = 1$};
\draw (3.5,2.3) node[anchor=north west] {$\triangle C = a_1$};
\draw (0.28,3.5) node[anchor=north west] {$C = a_0 + a_1 Y$};
\draw (0.5,1.1) node[anchor=north west] {Задача $\triangle C = a_1???$};
\end{axis}
\end{tikzpicture}
\caption{Изменение потребления}
\end{center}
\end{figure}
\end{document}