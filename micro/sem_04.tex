\documentclass[12pt,a4paper]{article}
\usepackage[14pt]{extsizes}
\usepackage[utf8]{inputenc}
\usepackage{amsfonts}
\usepackage{amssymb}
\usepackage{cmap}
% for fonts
    \usepackage[T2A, T1]{fontenc}
    \usepackage[english, russian]{babel}
    \usepackage{fontspec}
    \defaultfontfeatures{Ligatures=TeX,Renderer=Basic}
    \setmainfont[Ligatures={TeX, Historic}]{Times New Roman}
    \setsansfont{Times New Roman}
    \setmonofont{Courier New}
%plot
%mathcha.io
\usepackage{amsmath}
\usepackage{tikz}
\usepackage{mathdots}
\usepackage{yhmath}
\usepackage{cancel}
\usepackage{color}
\usepackage{siunitx}
\usepackage{array}
\usepackage{multirow}
\usepackage{amssymb}
\usepackage{gensymb}
\usepackage{tabularx}
\usepackage{booktabs}
\usetikzlibrary{fadings}
%mathcha.io
\usepackage{mathrsfs}
\usetikzlibrary{arrows}
\pagestyle{empty}
%plot
\usepackage{float}% for \begin{figure}[H]
\usepackage{cases}
\usepackage{graphicx}
\usepackage[left=2cm,right=2cm,top=2cm,bottom=2cm]{geometry}
\author{Аверьянов Тимофей, Корякин Алексей}
\begin{document}
\begin{center}
\textbf{План}
\end{center}

\begin{enumerate}
\item Структурная форма модели Маршалла-Вальраса;
\item Функция косвенной полезности;
\item Проверка $\displaystyle \boxed{\text{ДЗ}}$
\end{enumerate}

	На прошлом занятии мы обсудили понятие функции полезности. И отметили её свойства. Сейчас это понятие мы привлечём в процессе обсуждения модели поведения потребителя. Суть этой модели следующая:

	Потребитель приобретает такой набор благ: $\displaystyle \vec{x}^{*} \ =\ \left( x^{*}_{1} ,\ \dotsc ,\ x^{*}_{n}\right)$, который с одной стороны ему максиммально полез, а с другой стороны по карману (модель Маршалла-Вальраса):


\begin{equation*}
\begin{cases}
u\ =\ u( x_{1} ,\ \dotsc ,\ x_{n}) \ \rightarrow \max\\
{\displaystyle \sum ^{n}_{i=1} p_{i} x_{i} \ \leq \ M}\\
x_{1} \ \geq \ 0,\ \dotsc ,\ x_{n} \ \geq 0
\end{cases}
\eqno(1)
\end{equation*}
Экзогенными переменными в этой модели Маршалла-Вальраса являются:
\begin{equation*}
\left( M\ -\text{бюджет} ,p_{1} ,\ \dotsc ,\ p_{n} \ \left( -\ \text{цена}\right)\right) \eqno(2)
\end{equation*}
Эндлогенными переменными в этой модели Маршалла-Вальраса являются наборы благ:


\begin{equation*}
\vec{x} \ =( x_{1} ,\ \dotsc ,\ x_{n})
\end{equation*}
 Модель (1) служит примером задачи математического программирования на условный экстремум (Смотри lec\_01).

	К приведённой форме модель (1) трансформируется методом Лагранжа:
\begin{enumerate}
\item Составляется функция Лагранжа: $\displaystyle L=\ u( x_{1} ,\ \dotsc ,\ x_{n}) \ +\ l\left( M\ -\ {\displaystyle \sum_{i} p_{i} \ x_{i}}\right)$
\item Cоставляется необходимое условие экстремума:
\end{enumerate}
\begin{equation*}
\begin{cases}
\displaystyle{\frac{\partial L}{\partial x_{i}}} \ =0;\\[10pt]
\displaystyle{\frac{\partial L}{\partial l}} \ =\ 0;\\[10pt]
i\ =\ ( 1,\ \dotsc ,\ n)
\end{cases}
\eqno(3)
\end{equation*}
	3. Эти условия представляют систему $\displaystyle n+1$ уравнений с $\displaystyle n+1$ переменной.

	Система (3) решается либо аналитически, либо численно. Итогом решения является: $\displaystyle \vec{x}^{*} =\vec{x}^{M-B} \ ( M,p_{1} ,\ \dotsc ,\ p_{n})$ и множитель Лагранжа $\displaystyle l\ =\ l\ ( M,p_{1} ,\ \dotsc ,\ p_{n})$.

\textbf{Задача.}

Пусть функцией полезности потребителя служит логарифм Бернулли в ситуации двух благ эта функция имеет уравнение:


\begin{equation*}
u\ =a_{1} \ ( =\ 0.1) \ \cdotp \ \ln x_{1} \ +\ a_{2}( =\ 0.2) \ \cdotp \ \ln \ x_{2} \
\end{equation*}
Дано:

$\displaystyle \vec{x} \ =\ \left( x_{1} \ \left( =\text{молоко}\right) ,\ x_{\ 2} \ \left( =\text{хлеб}\right)\right)$

$\displaystyle M\ =\ 200$

$\displaystyle p_{1} \ =\ 50\ \text{p/кг}$

$\displaystyle p_{2} \ =\ 75\ \text{p/л}$

Найти:

Набор благ который приобретёт потребитель.


\begin{gather*}
u\ =a_{1} \ \cdotp \ \ln x_{1} \ +\ a_{2} \ \cdotp \ \ln \ x_{2} \ +l( M\ -\ ( 50x_{1} \ +\ 75x_{2})) =\\
= a_{1} \ \cdotp \ \ln x_{1} \ +\ a_{2} \ \cdotp \ \ln \ x_{2} \ +\ Ml\ -50x_{1} l-75x_{2} l
\end{gather*}

\begin{equation*}
\begin{cases}
\frac{\partial L}{\partial x_{1}} \ =\frac{a_{1}}{x_{1}} \ -\ 50l\ =\ 0;\\
\frac{\partial L}{\partial x_{2}} \ =\ \frac{a_{2}}{x_{2}} \ -\ 75l\ =\ 0;\\
\frac{\partial L}{\partial l} \ =\ M\ -\ 50x_{1} \ -\ 75x_{2} \ =\ 0;
\end{cases}
\end{equation*}

\begin{equation*}
M\ -\ \frac{50a_{1}}{50l} \ -\ \frac{75a_{2}}{75l} \ =\ 0
\end{equation*}

\begin{equation*}
M\ =\ \frac{a_{1}}{l} \ +\ \frac{a_{2}}{l}
\end{equation*}

\begin{equation*}
Ml\ =\ a_{1} \ +\ a_{2}
\end{equation*}


	Таким образом по правилам (5) рассчитывается спрос потребителя по Маршалу-Вальрасу. Доведём до чисел:


\begin{equation*}
l\ =\ \frac{a_{1} \ +\ a_{2}}{M} \ =\ \frac{0.3}{200} \ =\ 0.0015 \eqno(4)
\end{equation*}



\begin{equation*}
x^{*}_{1} \ =\ \frac{a_{1} \ M}{( a_{1} +a_{2}) p_{1}} \ =\ \frac{0.1\ \times \ 200}{0.3\ \times 50} \ =\ 1.3 \eqno(5)
\end{equation*}

\begin{equation*}
x^{*}_{2} \ =\ \frac{a_{2} \ M}{( a_{1} +a_{2}) p_{2}} \ =\ \frac{0.2\ \times \ 200}{0.3\ \times 75} \ =\ 1.77 \eqno(5')
\end{equation*}
Рассчитаем значение функции полезности:


\begin{equation*}
u\ =\ a_{1} \ \ln x_{1} \ +\ a_{2} \ \ln x_{2} \ =\ 0.1\ \ln \ 1.3\ \ +\ 0.2\ \ln 1.77\ =\ 0.14
\end{equation*}
\textit{Замечание.} Подстановка вектора $\displaystyle \overrightarrow{x^{*}}$ превращает эту функцию:
\begin{equation*}
u\ =\ u\left( x^{*}_{1} ,\ \cdots ,\ x^{*}_{n}\right) \ =\ u( M,\ p_{1} ,\ \cdots ,\ p_{n}) \ \eqno(6)
\end{equation*}
в функцию экзогенных переменных. Экономисты называют эту функцию \textit{функцией косвенной полезности} потребителя, в нашем примере значение этой косвенной функции полезности оказалось равной значению 0.14.

$\displaystyle \boxed{\text{ДЗ}}$ рассчитать спрос потребителя по модели Маршала-Вальраса принимая в качестве функции полезности неоклассическую функцию:
\begin{equation*}
u\ =\ a_{0} \ ( a_{0} \ =1) \ \cdot \ x^{a_{1}}_{1} \ \cdot \ x^{a_{2}}_{2} \eqno(6')
\end{equation*}
Параметры этой функции и значения переменных принять такими же как в аудиторных задачах.
\begin{center}
\textbf{Свойства спроса потребителя по модели \ Маршала-Вальраса}
\end{center}
Показать что спрос потребителя остаётся неизменным, если его бюджет и цены благ изменяются одновременно в некоторое кол-во $\displaystyle k$.

Доказательство привести с помощью формулы (5):
\begin{equation*}
x^{*}_{1} \ =\frac{a_{1} \ Mk}{( a_{1} +a_{2}) p_{1} k} \eqno(7)
\end{equation*}

\begin{equation*}
x^{*}_{2} \ =\ \frac{a_{2} \ Mk}{( a_{1} +a_{2}) p_{2} k} \eqno(7')
\end{equation*}
Следовательно спрос не изменится. Свойство (7) остаётся справедливым для любой функции полезности. Математики называют такое свойство \textit{свойством однородности нулевой степени по Маршалу-Вальрасу}.

$\displaystyle \boxed{\text{ДЗ}}$ Показать, что и для неоклассической функции свойство однородности сохраняется.


\begin{equation*}
u\ =\ a_{0} \ ( a_{0} \ =1) \ \cdot \ x^{a_{1}}_{1} \ \cdot \ x^{a_{2}}_{2}
\end{equation*}
\textbf{Задача}

Вычислить экономический смысл множителя Лагранджа ($\displaystyle l$). (Заглянем в 1 и 2 занятие).

\textbf{Решение:}

Вернёмся к выражению
\begin{equation*}
u\ =a_{1} \ \cdotp \ \ln x_{1} \ +\ a_{2} \ \cdotp \ \ln \ x_{2}
\end{equation*}
 и с учётом выражения (5) получим выражение этой функции:


\begin{equation*}
u^{*} =a_{1} \ \cdot \ln\left(\frac{a_{1} \ M}{( a_{1} +a_{2}) p_{1}}\right) \ +a_{2} \ \cdot \ln\left(\frac{a_{2} \ M}{( a_{1} +a_{2}) p_{2}}\right) \eqno(8)
\end{equation*}
$\displaystyle \boxed{\text{ДЗ}}$ Можно показать, что предельное значение $\displaystyle y^{*}$ по бюджету потребителя в точности равно множителю Лагранджа.

Дадим трактовку множителю $\displaystyle l$. $\displaystyle l\ -$ \ дополнительная полезность по Маршалу-Вальрасу, которая возникает в ответ на дополнительную еденичу денежных средств ($\displaystyle M$).
\end{document}
